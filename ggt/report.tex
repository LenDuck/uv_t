\documentclass[a4paper]{article}
\usepackage[T1]{fontenc}
\usepackage[utf8]{inputenc}
\usepackage{lmodern}
\usepackage{listings}
\usepackage[english]{babel}
\lstset{breaklines=true}
\author{Leendert van Duijn\\6143997\\Robert Rinia\\0000000}
\title{GoogleTranslateCLI}
\begin{document}
\maketitle
\tableofcontents
\newpage
\section{Goal}
The goal we had in mind was a simple yet effective application to translate
snippets of text using GoogleTranslate. A simple and effective tool to provide
quick translations.
\subsection{Features}
In order to obtain our goal we need several features:\\
Being able to choose the language to translate from and to\\
Being able to read a longer text, multiline of course.
\subsection{Interface}
Something simple and effective, a command line interface capable of being called
within scripts without any problems. Simply asking for all the unknown
parameters where there is the option of pasing them as arguments to the program
for automated translation.

\section{Implementation}
The process of getting a translation from the Google Translate service consists
of several parts\\
Obtain a text and languages.\\
Form a request using this information and a key.\\
Send the request to the google translate service.\\
Receive a response in a fixed format.\\
Decipher the response and return or print the result.\\
\newline
\subsection{Chosen structure}
The functionality needed is implemented in several pieces, this to ease
debugging of parts of the process and enable some adaptability. These features
are bundled together:\\
Getting input and forming a request.\\
Sending the request and receiving the response using https.\\
Parsing the output in order to obtain the result to print.\\
\subsection{Language and libraries}
We have chosen to use the C programming language, in combination with some smart
use of bash/sh scripting to tie all parts together. We have opted to use a
separate program for the network connectivity, the curl program which can
easily be used for sending specific http requests even over a secure
channel\footnote{https, which uses ssl}. 
\section{Trouble}
We had an initial problem in connecting to Google, which is due to the ssl layer
and a missing certificate, this problem is alleviated by passing curl the
argument telling it to accept a possibly invalid server
certificate\footnote{Which could be a problem when dealing with sensitive
information as the identity of the 'other' computer is not verified, but this is
a utility to ask Google to translate something\ldots not a login page for a
bank}. We also had some trouble concerning "bad requests" where we forgot to
escape several characters which may not be in an URL. The Google API mentions a
limit of 1500 characters in the URL\footnote{This could be a http imposed
limit}, this limits our request size to about 1300 characters which could be
improved to about 3300 character which is possible by using a POST request
instead of the GET requests, however this changes the limit from about 1 page of
text to 2 pages of text. So splitting is still needed when you want to translate
any decent length of text. We have chosen to honour the GET limit and enable you
to split your requests into acceptable chunks.
\section{Conclusion}
Our application is simple yet effective and in our translation tests fully
functional, the script/frontend can be run without any parameters to be
interactive, passed an argument --help for some guidance, or the languages to be
used in translation. The last parameter is the file to read the input from, if
anything other than stdin is to be used. Concerning the language choice, please
consult google for this information as this list is out of our control. 

The following calls can be made:\\
frontend\\
frontend --help\\
frontend FROM\\
frontend FROM TO\\
frontend FROM TO SOURCEFILE\\
\appendix
\section{Code}
\subsection{Frontend}
This is the code needed to tie it all together neatly.
\lstinputlisting[language=bash]{./frontend.sh}

\subsection{Defines}
The central place to define the request size limit.
\lstinputlisting[language=C]{./defines.h}

\subsection{Formrequest}
This forms a valid request from asked or supplied input.
\lstinputlisting[language=C]{./formrequest.c}

\subsection{Soapscan}
This filters its input to only show interesting data.
\lstinputlisting[language=C]{./soapscan.c}

\end{document}
