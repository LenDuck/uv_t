\documentclass[a4paper]{article}
\usepackage[T1]{fontenc}
\usepackage[utf8]{inputenc}
\usepackage{lmodern}
\usepackage{listings}
\usepackage[english]{babel}
\lstset{breaklines=true}
\author{Leendert van Duijn\\6143997\\Robert Rinia\\6076777}
\title{Chat server/client}
\begin{document}
\maketitle
\tableofcontents
\newpage
\section{Goal}
The goal for this assignment was simple: ``Make a chat server/client pair that
supports the following:\\

\begin{enumerate}
\item \label{1}Works in accordance with the given protocol.
\item \label{2}Runs on the UvA linux machines.
\item \label{3}Server supports multiple clients.
\end{enumerate}

\subsection{Features}
In order to obtain our goal we need several features:\\
\begin{itemize}
\item Requirement No. $2$ disables us from using MS Visual Studio, so we need
to choose an other environment.
\item Requirement No. $3$ is achievable by using threads.
\end{itemize}

\section{Implementation}
\subsection{Language}
For this assignment, we chose to work with the language C, because that
language doesn't require the user to type a lot of `useless' keywords, it has
less overhead and it supports everything we thought to be needing `out of the
box'.

\subsection{Architecture}
For this assignment we chose to do our own memory management (that's also a
reason we did it in C) by treating our pointers in a smart way so that there
are no memory leaks and such.

We also made every client have its own server thread in order to reduce the
chance of deadlock and the time a client has to wait for a server response.

We have used an Object oriented approach for some things, most notably the
connection code (con\_t) which is used in both our client and server for sending
and receiving messages. It supports both bitwise (raw) and
line-wise sending of data, though for our implementation we only used the
line-wise variant because we knew in advance that only ASCII would be sent over
the connection.
\subsection{Client interface}
We'll need an interface without any unnecessary ``features'' like GUIs or TUIs,
because they only distract from the \emph{real} power that is automatic parsing
and recognition of commands, even if there are made mistakes in the CaSE oF tHe
ComMaNds.

\subsection{Chosen structure}
The server threads are:
\begin{itemize}
\item main, this thread always runs until network problems occur.
\item accept, this thread is used for creating new client threads and is
referenced by the main thread when a client tries to connect.
\item client thread, these threads (one for every \emph{connected} client)
handle the client/server-interaction after a client has connected. They wait
with infinite patience for their respective clients to send something so that
we were able to test our server even without a client using netcat. Once the
thread receives a message it might change its state and tell the main thread
about it.
\end{itemize}

Please note the server is able to send data to all/some/one client from
\emph{every} thread by using mutex-locks in a smart and efficient way.

Here follows a list of our files and what they do:
\begin{itemize}
\item `./src\_server/main.c' : the main server thread.
\item `./src\_server/global\_server.c' : the accept thread.
\item `./src\_server/client\_handling.c' : the per-client thread.
\item `./src\_server/message.c' : the message handling thread.

\item `./src\_client/main.c' : contains almost everything used in the client,
except connection-related things..

\item `./src\_con/con.c' : handles everything connection-related.

\item `./src\_dlogger/dlogger.c' : the logger.
\end{itemize}

\section{Conclusion}
We have made a fully working server/client couple which has shown to work even
in our crosstests with other clients/servers.

The client can be run without arguments, assuming the server is at localhost
port 55555, or with an IP-address and a port number as arguments so it can also
connect to other computers than the one the server is on (which is quite
useful).

The following calls can be made:
\begin{itemize}
\item ./chat\_client
\item ./chat\_client IP PORT\_NUMBER
\end{itemize}

\appendix
\section{Code}
\subsection{Server}
\subsubsection{Main}
The ``main thread''.
\lstinputlisting[language=C]{../src_server/main.c}

\lstinputlisting[language=C]{../src_server/server.h}

\subsubsection{Client handing}
This handles the clients.
\lstinputlisting[language=C]{../src_server/client_handling.c}

\subsubsection{Global server}
The ``accept thread''.
\lstinputlisting[language=C]{../src_server/global_server.c}

\subsubsection{Message handler}
Parses a string into a message, handles the message, then sends a message
(which is parsed back into a string first) according to the protocol.
\lstinputlisting[language=C]{../src_server/message.c}

\lstinputlisting[language=C]{../src_server/message.h}

\subsubsection{Type definitions}
Contains all data types used in the server.
\lstinputlisting[language=C]{../src_server/types.h}

\subsection{Client}
\subsubsection{Main}
The heart of the client. Does practically everything not connection-related.
\lstinputlisting[language=C]{../src_client/main.c}

%\lstinputlisting[language=C]{../src_client/client.h}

\subsection{Connection}
This takes care of the connection. It is able to establish a connection, send
and receive data and close a connection afterwards.
\lstinputlisting[language=C]{../src_con/con.c}

\lstinputlisting[language=C]{../src_con/con.h}

\subsection{Logger}
This logs almost everything that happens in the connection.
\lstinputlisting[language=C]{../src_dlogger/dlogger.c}

\lstinputlisting[language=C]{../src_dlogger/dlogger.h}

\end{document}